\documentclass[10pt, a4paper, titlepage]{jarticle}

\usepackage{amsmath,ascmac}
\usepackage{bm}
\usepackage{algorithm,algorithmic}
\usepackage{listings}
\usepackage{here}
\usepackage[subrefformat=parens]{subcaption}
\usepackage[dvipdfmx]{graphicx}
\usepackage{amsmath, amssymb}
\usepackage{type1cm}
\usepackage{url}
\usepackage{longtable}

\lstset{%
  language={Python},
  basicstyle={\small},%
  identifierstyle={\small},%
  commentstyle={\small\itshape},%
  keywordstyle={\small\bfseries},%
  ndkeywordstyle={\small},%
  stringstyle={\small\ttfamily},
  frame={tb},
  breaklines=true,
  columns=[l]{fullflexible},%
  numbers=left,%
  xrightmargin=0zw,%
  xleftmargin=3zw,%
  numberstyle={\scriptsize},%
  stepnumber=1,
  numbersep=1zw,%
  lineskip=-0.5ex,%
}
\renewcommand{\lstlistingname}{コード}
\renewcommand{\lstlistlistingname}{コード目次}
\newcommand{\fracd}[2]{\frac{\mathrm{d} {#1}}{\mathrm{d} {#2}}}
\makeatletter
\renewcommand{\theequation}{%
\thesection.\arabic{equation}}
\@addtoreset{equation}{section}
\makeatother
\usepackage{ascmac}
%
% 以後をレポート課題ごとに書き換える。
%

\begin{document}
\title{数理工学実験 \\テーマ:最小二乗法}
\author{2回生 \quad 田中風帆(1029321151) \\実施場所:自宅}
\date{実施:2021年12月13日\\提出:2021年12月日}

\maketitle
\setcounter{page}{0}\pagenumbering{roman}
\tableofcontents
\clearpage
\listoffigures
\clearpage
\listoftables
\clearpage

\part{概要}
\pagenumbering{arabic}
このレポートでは、最小二乗法の関わる課題を解いた。重回帰問題、
多項式回帰問題を解き(課題8、9)、観測誤差の分布やデータの取り方によって
結果にどのような違いが出るかを確認し、その理由を考察した(課題10、11)。観測値の次元
が多次元になる場合についても同様のことを行い(課題12)、観測誤差の分布が異なるデータを
組み合わせたものに対しても回帰を行った(課題13)。データを分割してパラメータを推定し、そ
れらを合成することで、全データを使用した場合と同様の推定値を得られることを確認した(課題14)。
さらに偶然誤差の分散の推定値を求め、それを用いてパラメータの推定値を合成し、全データを用いた
回帰により得られたパラメータとどちらが優れているか比較、考察した(課題15)。バネマスダンパ系
を表す直線状の運動方程式のパラメータを逐次最小二乗法で求めた(課題16)。この際、正則化項など
に工夫を加えた。正弦波で変動する信号から得られたデータに対して忘却係数つきの逐次最小二乗法
で回帰を行い、各時刻でパラメータを推定した(課題17)。Kalmanフィルタを実装し、与えられた離散
時間ダイナミクスおよび観測方程式に適用することで時間ごとのパラメータの推定値を求めた(課題18)。
RTSアルゴリズムを実装し、課題18の設定のもとでパラメータの初期値を推定し、またその推定誤差分散を
求めた(課題19)。交互最小二乗法を実装し、与えられたデータに適用することでパラメータを推定した。
また、初期値を変えて同様のことを繰り返し行った(課題20)。最後に、与えられたデータにk-means法を適用し、
3つのクラスターに分類した。異なる初期値に対して、同様のことを繰り返し行った。
\clearpage
\part{課題1}


\end{document}